\documentclass{ctexart}
\usepackage[hidelinks]{hyperref}
\usepackage{geometry}  
\geometry{a4paper,left=25mm,right=20mm,top=25mm,bottom=25mm}
\usepackage{amsmath, amsthm, amssymb, appendix, bm, graphicx, hyperref, mathrsfs}
\usepackage{lastpage} %总页数
\usepackage{fancyhdr} %使用fancyhdr
\pagestyle{fancy}                   % 设置页眉页脚
\lhead{page \thepage\ of \pageref{LastPage}}   %页眉左侧显示页数                 
\chead{丝绸之路文学史话}                                  %页眉中
\rhead{\small\leftmark}                         %章节信息                       
\cfoot{\thepage}                                %当前页,记得调用前文提到的宏包                
\renewcommand{\headrulewidth}{0.1mm} %页眉线宽,设为0可以去页眉线
\renewcommand{\footrulewidth}{0.1mm} %页脚线宽,设为0可以去页眉线


\title{\textbf{日本神话与文化的形成}\\{\small --以《古事记》为例}}
\author{计算机2101 陈实}
\date{\today}
\linespread{1}
\newtheorem{theorem}{定理}[section]
\newtheorem{definition}[theorem]{定义}
\newtheorem{lemma}[theorem]{引理}
\newtheorem{corollary}[theorem]{推论}
\newtheorem{example}[theorem]{例}
\newtheorem{proposition}[theorem]{命题}
\renewcommand{\abstractname}{\Large\textbf{摘要}}

\begin{document}
\setlength{\baselineskip}{20pt}
\maketitle

\setcounter{page}{0}
\maketitle
\thispagestyle{empty}

\begin{abstract}
    \setlength{\baselineskip}{20pt}
    神话与神道信仰是日本思想史研究中的重要课题,而《古事记》作为神道原
典,是该领域研究中最重要的文献资料之一。《古事记》融汇了神话传说、历史记事等多种内容,囊括了古代日本社会的各色面貌,对日本文化
的形成与发展产生了深远的影响。本文将以《古事记》为例,探讨日本神话与文化的形成过程,以及《古事记》在日本文化史上的地位与作用。
    \par\textbf{关键词:}日本神话; 日本文化;《古事记》; 
\end{abstract}

\newpage
\pagenumbering{Roman}
\setcounter{page}{1}
\tableofcontents
\newpage
\setcounter{page}{1}
\pagenumbering{arabic}

\section{《古事记》的成书与流传}

\subsection{《古事记》的成书}

《古事记》作为现存日本最
早记录先代诸事的史学文献之一,全篇由三卷构成。上卷记载序文、神代传说,
中卷和下卷分别记载神武天皇至应神天皇、仁德天皇至推古天皇时期的天皇帝纪
和神代旧辞。据《古事记》序文记载,奉元明天皇之命,稗田阿礼咏诵“帝皇日
继”“先代旧辞”等先代口传故事和历朝帝王事迹,太安万侣“撰录稗田阿礼之
勅语旧辞以献上”,于和铜五年(712 年)正月二十八日,编纂成书。

\subsection{《古事记》的成书背景}

5世纪末,日本武烈、继体、安闲、宜化天皇相继继位,围绕皇统继承问题,
皇权内部派阀相争,政局分裂。这一时期,朝鲜半岛新罗、高句丽势力扩张,人
员流动势必带来文化的交融,佛教与中原文化在日本影响日增,本土文化与外来文化之间的问题开始明显
。\par
大化改新后,日本政治、经济、文化得到了充足的发展,在此阶段,修订诸家旧记,编纂国史,成为稳固天皇政权,强调政治合法性的重
要环节。是故文中特别强调“皇权正统”和“日本独立”的思维,深度体现了日本欲独立于中国、朝鲜的史观和主张。

\subsection{《古事记》神代卷}

在七世纪时,日本神话体系的主体内容逐渐形成,融合了先代氏族信仰,并汲取了东亚广泛的民俗文化。
这一古老而复杂的神话传承汇聚了各家各派的元素和错综复杂的仪式。这些文献中的神话叙事共同构建了日本神话世界的三大领域,
分别以天照大神为核心的天孙系神话,以大国主神为中心的出云系神话,以九州日向为中心的筑紫系神话而闻名。这些神话舞台和众
神形象共同诉说了日本的起源与皇室的由来,也为后来神道祭祀神和民间信仰神提供了共同的祖源。

“二元观”是神代卷的基本特征之一,体现在神话中的大和和出云这两个根本差异的概念上。大和系属于天神系,是天孙降临的统治者,
而出云系国神则位于地上界的被支配者,这种二元观呈现出明显的对立性。因此,古代人的世界观中将大和系视为“显世”的地上界,
而将出云系隐藏在“幽冥”的地下界。从这个角度出发,神代卷将世界原初状态分为天界高天原、地界苇原中国、地下界黄泉国或根之
坚洲国这三层结构。

\subsubsection{天孙系神话}

《古事记》序文有载:“混元既凝,气象未效,无名无为,谁知其形。”
即世界起初是一片混沌,国土尚以一种油脂状态漂浮于空间内,其后混沌中诞生的
苇牙萌腾之物升腾为“天”,余下部分化为“地”,接着依次生成了别天神、神
世七代。在此混沌世界诞生的伊邪那岐、伊邪那美二神受命携天沼矛修理固成多
陀用币流之国,降于天浮桥,以矛头搅动混沌的世界,矛头滴落之盐水化为岛屿。
其后二神又降于八寻殿,立天之御柱,绕柱成婚生万物。\par

这相当于日本神话中的创世纪,是日本神话的起源。中国古籍中《幼学琼林》中“轻清者上浮而为天,重浊者下凝而为地”,
可以看出,日本神话受到中国的神话的影响很深。古事记》中的高天原是神话构想中众神的领土,与人生存的国土苇原中国
相对应,高天原居住八百万众神,位于核心位置的是天皇家的祖先神天照大御神,
以高天原为舞台发生的故事被称为高天原神话。这里提到的八百万众神,是日本神话中的一个重要概念,它奠定了
日本文化中万物有灵的基础,对当今日本人对于自然的敬畏和对于万物的尊重,都有着深刻的根源。

\subsubsection{出云系神话}

出云神话就是在此出云地发生的神话故事,这些神话以建速
须佐之男命与大国主神神话为核心。出云地区因其地缘优势自古就是日本政治、文化、宗教的核心区域,《出云
国风土记》中将此地称为“神的故乡”。一说认为,出云继承了绳文时期的自然崇拜信仰,是日本精神文化的发祥地。


\subsubsection{日向系神话}
日向神话最具代表的是《古事记》中天孙迩迩艺命承诏
天照大御神的神谕,从高天原降于筑紫日向袭国的高千穗峰,治理苇原中国。此
外,还包括以民间故事浦岛太郎为原型的“海幸山幸”传说,以及叙述初代神武
天皇自日向出发征服大和国的“神武东征”神话。这三大神话的集合被称为“日
向三代”神话。并且,日向神话中的神话叙事主体由“神皇”转为“人皇”。学界普遍认为,神
武东征是为了解释皇权起源及其合法性问题而刻意美化了的神话,讲述了九州势
力迁至大和,建立大和王权的神话叙事。

\section{《古事记》对日本文化的影响}

\subsection{日本神道教的产生}

文字是文化的载体,在中华文明的发展、传承过程中,汉字不
断向周边地区扩散,对东亚国家的文化发展起到了尤其关键的作用。《古事记》
的形成在日本文化史上具有里程碑式的意义,当民族口传记事转为笔录文字时,
日本先民逐步改变了以往以直觉、体验为主的思维模式,在吸收、学习汉字背后
所蕴含的理性思维和逻辑的同时,大量保存了中国文化,包括价值观念、民俗传
统等,使得中日文化间存在诸多相似之处。因此,文字不仅是一种语言符号,通
过文字记录,日本先民世代传承的古神话传说、民俗仪礼、上古歌谣等都得到了
较为完整的保存。\par
而在中国文化流入日本的过程中,道家思想对日本神道的形成产生了不可忽
视的影响。道家是我国先秦时期的重要流派之一,其思想核心是认为“道”构成
了宇宙万物的本源、事物的机理以及人与社会的运行法则。日本道教研究者福永
光司认为,《古事记》神代卷作为神道教的重要典籍,其内容与道教思想有很深
的关联,他还明确指出作为日本古代宗教信仰核心的天皇信仰也深受道教的影
响。 可以发现,由于日本古传信仰是建立在自然崇拜、祖先崇拜、神灵崇拜的
基础上的,因此其在借用汉字对世界起源、生命诞生等问题做出解释时,也吸收
并继承了文字所承载的文化内涵与思维模式。

\subsection{日本的生死观}

《古事记》神话所展现的对于生命和死亡的理解,最初是基于日神信仰形成的。
神道思维中神并非永生,同样需要面对死亡,甚至在神话叙事中表
达了神面对死亡时具有恐惧、厌恶的伦理意识,并将这种观念与认识通过隐喻,
以“祓”的方式表达其排斥“罪秽”、拒绝死亡的宗教心理。这种生死观念带来
了两方面的影响。\par
其一,“触秽”意识导致神道祭祀形成多重禁忌,对现世的人
形成约束。神道教义将“秽”等同于“罪”,破坏农耕祭祀、违背神道意志被定
义为“天津罪”,天灾或个体的犯罪被称为“国津罪”。这种观念导致了日本产生了“耻文化”,在多神教的日本,强烈意识到的是世人的眼光。不是靠正确与否决定行动,而是凭借世人怎么想来决定自己的行动,这就是耻文化。耻文化的出现与日本的社会构造有深刻的联系。

耻文化对日本人的影响:在耻文化的风潮下,个人的品行是高雅还是卑劣,行为是都正确,这些全靠他人来判断。所以日本人的行为原则是只要推测别人是怎么判断的,就以他人的判断为基准,确定自己的行动方针。\par
其二,神道的生死观念导致死生世界分为多层“他界”,“黄泉”“常世”既代表了神道思想中摆脱世俗,
实现永生的美好愿望,又体现了神道观念中超然物外的思想境界。与中国不同的是,死后的“黄泉”或“常世”并没有同现实世界相断绝,而是死后人的灵
魂化为神灵存续于人间

神道通过强化死后灵魂将进入净土的观念,从而淡化生死、宽容生死。
死后,人的生命转化为一种灵魂的形式,也就是变为先祖。伊藤干治认为古代日
本人的意识中普遍存在一种倾向,即把现世与异界、生与死之间做同一次元化理
解。换言之,生与死之间没有明显的割裂,而被作为一个连续的概念看待。在这
种观念之中,死并不与生对立,而是成为了生的一部分。这也正是李泽厚总结
的日本人“重生安死”“惜生崇死”的独特生死观。这种特殊的“生死”观念直
接或间接地导致日本人的审美和伦理意识同周边民族产生差异,其对待死亡产生
的一种唯美化倾向,甚至成为日本精神世界“物之哀”特质的重要部分

\subsection{日本右翼思想的形成}

近代,神道被皇权政治恶用,在国家意志主导下逐渐沦为政治
神学的一部分。为增强本民族文化认同,神道以颠覆传统的姿态片面贬低儒佛等
外来思想的价值,认为其削弱甚至抹杀了日本古来固有的神道理想,故致力于发
扬外来思想流入以前的所谓日本民族精神。这种复古、排他的主张受到一部分国
学者的追捧,他们标榜日本固有文化、精神的意义,借国学为日本导入了一个新
的价值参照系统;此外,还出现了主张依据“记纪”本意解释神道思想的复古神
道流派,二者共同推动了中世以来日本试图复归民族、国体“本真”的实践。不
过,这一系列主张在根源上是建立在贬斥文化母体、强调“自民族中心主义”的
心态之上的,真实反映了千百年来日本隐忍于内心深处的亟欲实现“民族自决”
的心理。\par
甲午一役,东亚格局骤变,日本迅速重新审视中日关系,并以“突进”的姿
态试图摆脱日渐衰颓的东亚秩序,走上“脱亚入欧”之路,与中国展开“全新”
交往。其具体做法有利用神话中的“原罪”思想强化宗教权威、将神的“超越性”
赋予天皇神格与神体、以“天孙降临”“神武建国”等“神国”叙事作为日本超
越法理界限的依据等。在此期间,政治统摄下的日本神灵信仰以自体膨胀的方式
巧换装容,谋求建立全新的价值体系,这一神道“理想”构成了近代日本扩张史
中近乎不变的底色。

\subsection{当代日本文化输出中的神话元素}

神话传说题材,作为电视动画的重要表现领域,
具有猎奇性、趣味性、文化属性,题材类型等方面的优
势,受到世界各国观众的青睐。日本以“神道教”“怪
谈”等文化传统为创意起点,创作出诸如《夏目友人
帐》《犬夜叉》《虚构推理》等优秀电视动画作品。


神道教作为日本民族的本土宗教,发端于日本先
民对于自然的原始崇拜,他们崇尚万物有灵,奉行泛神
主义,认为天地万物皆可为神。因此,在这一思想的影
响下,日本神话题材动画中出现的神鬼妖怪等角色,形
形色色且包罗万象。《夏目友人帐》的这一特征十分明
显,不仅有动物成精,植物成妖,茶杯、萤火虫和行将
枯萎的藤蔓也可幻化成形。在作品《犬夜叉》中亦是如
此,动物、植物、能剧的表演面具、笛子、还有被戈薇
怨念所驱使的陶俑。神道教的泛神论主义对电视动画创
作者们的影响巨大,给予他们创作的灵感与养分。

植根于神道教传统对神的信仰,神职人员也频频出现在这一
题材的动画作品中。例如巫女、神主、神使、神官等,
《犬夜叉》中戈蓝的身份即是“巫女”,《虚构推理》
中身体残缺的女主岩永琴子,亦是妖怪的智慧之神及仲
裁和解决它们之间纠纷的“巫女”形象,电影《你
的名字》中女主宫水三叶,在村中神社的祭祀活动中也
承担着“巫女”的职责。

中国的古代神话比日本的神话更加丰富,本可以为中国的动画创作者提供更多的灵感,
但是中国的动画创作者却没有好好利用这一资源,创作题材集中在《西游记》《封神演义》等少数几部神话故事上,
并且将时代背景定格在古代,对故事的讲述也显得按部就班,不敢跳出窠臼,没有够将神话故事与现代生活相结合,创作出更多更好的动画作品,
这是中国动画创作者值得反思的地方。

\section{结语}

《古事记》作为现存日本最早记录先代诸事的史学文献之一,以其三卷的形式承载着丰富而深刻的神话传承。这部古老的文献不仅记录了日本神话的起源和发展,也深刻地反映了日本文化的演变和与外部文化的交融。

在成书背景中,我们见证了日本社会在5世纪末的政治分裂和文化交流。大化改新后,为巩固天皇政权,修订旧记成为必要的举措。这一时期的《古事记》强调了“皇权正统”和“日本独立”的思维,反映了日本渴望独立于中国和朝鲜的史观。

神代卷展示了日本神话的多元性,涵盖了天孙系神话、出云系神话和日向系神话。这些神话构建了日本神话世界的三大领域,塑造了日本文化的起源和皇室的由来。而神话中的“二元观”和对生死的特殊观念,深刻地影响了日本文化的发展,形成了独特的生死观和对自然的敬畏。

《古事记》对日本文化的影响延伸至当代,尤其在文化输出中。神话传说成为电视动画的重要创作题材,展现了神道教的泛神论思想和对生命、死亡的独特理解。此外,神道教在日本右翼思想的形成中扮演了重要角色,为建立全新的价值体系提供了理论基础。

最后,在当代日本文化输出中,神话元素继续发挥着重要作用。电视动画以神道教为创意起点,呈现了丰富多彩的神话世界,传承和展示着日本文化的独特魅力。然而,与中国古代神话相比,日本在利用这一资源上表现更为积极,为中国动画创作者提供了启示和反思的空间。



\newpage

\begin{thebibliography}{99}
    \bibitem{a}司娟.从文化角度解析日本神话《古事记》 [J].济南职业学院学报,2012,(03):117-119.
    \bibitem{b}李芳.看《古事记》解析日本伦理意识[J].时代文学(下半月),2012,(03):73-74.
    \bibitem{c}王琛.日本神话中的道教思想研究[D].天津科技大学
\end{thebibliography}


\end{document}